\documentclass{unswmaths}

\usepackage{unswshortcuts}

\begin{document}

\subject{}
\author{}
\title{}
\studentno{}


\newcommand{\Real}{\operatorname{Re}}
\newcommand{\Img}{\operatorname{Im}}
\newcommand{\lan}{\langle}
\newcommand{\ran}{\rangle}
\newcommand{\Proj}{\mathbb{P}}
\newcommand{\isom}{\cong}
\newcommand{\id}{{\operatorname{id}}}
\newcommand{\ha}{\boldsymbol{m}}
\newcommand{\Circ}{\mathbb{T}}
\newcommand{\BMO}{{BMO}}
\newcommand{\sgn}{\operatorname{sgn}}
\newcommand{\Diff}{\mathcal{D}}
\newcommand{\pvint}{\mathrm{p.v.}\int}
\newcommand{\BMOA}{{BMOA}}

\section*{Introduction}
    A \emph{Hankel matrix} is an infinite matrix of the form $(a_{j+k}\;:\;j,k\geq 0)$ for some 
    sequence $(a_j\;:\;j,\geq 0)$.
    
    Such a matrix defines a linear operator on the space of sequences indexed by $\Ntrl$ of finite support,
    which we denote $\mathcal{F}$.
    
    The inverse fourier transform of $\mathcal{F}$ is the set of polynomials on $\Circ$,
    which we denote $\mathcal{P}$.
    
    The matrix $A = (a_{j+k}\;:\;j,k\geq 0)$ acts on the sequence $(x_k\;:\;k\geq 0) \in \mathcal{F}$ by matrix
    multiplication,
    \begin{equation*}
        (Ax)_j = \sum_{k\geq 0} a_{j+k}x_k.
    \end{equation*}
    
    These notes cover the conditions under which a Hankel matrix acting in this way extends to a bounded linear
    operator on $\ell^2(\Ntrl)$, and under what conditions this operator is in some ideal of
    compact operators.
    
    
    
\section*{Boundedness}
    Suppose that $A = (a_{j+k}\;:\;j,k\geq 0)$ is a Hankel matrix. We wish to find
    conditions on $(a_j\;:\;j\geq 0)$ such that $A$ extends to a bounded operator on $\ell^2(\Ntrl)$. 
    This problem is solved by Nehari % Add reference here 
    as follows:
    \begin{theorem}
        The Hankel matrix $A = (a_{j+k}\;:\;j,k\geq 0)$ extends to a bounded operator on $\ell^2(\Ntrl)$
        if and only if there is some $\psi \in L^\infty(\Circ)$ such that
        \begin{equation*}
            \hat{\psi}(n) = a_n
        \end{equation*}
        for $n\geq 0$.
%        Moreover,
%        \begin{equation*}
%            \|A\|_{\ell^2(\Ntrl)} = \inf\{\|\psi\|_{\infty}\;:\;\hat{\psi}(n) = a_n,n\geq 0\}.
%        \end{equation*}
    \end{theorem}
    \begin{proof}
        First assume that $\psi \in L^\infty(\Circ)$ with $\hat{\psi}(n) = a_n$ for $n\geq 0$.
        Suppose that $x = (x_j\;:\;j\geq 0)$ and $y = (y_j\;:\;j\geq 0)$ are in $\mathcal{F}$.
        Then we compute,
        \begin{align*}
            \langle Ax,y \rangle &= \sum_{j,k\geq 0} a_{j+k}a_j\overline{y}_k\\
            &= \sum_{j,k\geq 0} \hat{\psi}(j+k)x_j\overline{y}_k\\
            &= \sum_{n \geq 0} \hat{\psi}(n)\sum_{j = 0}^m a_j\overline{y}_{m-j}\\
            &= \sum_{n\geq 0} \hat{\psi}(n) (x * \overline{y})(n)\\
            &= \langle \hat{\psi},\overline{x}*y\rangle
        \end{align*}
        Now let $q = (\overline{x}*y)^{\vee} = \check{\overline{x}} \check{y}$. Hence,
        \begin{align*}
            \langle \hat{\psi},\overline{a}*b\rangle &= \langle \psi,q\rangle\\
            &\leq \|\psi\|_{\infty} \|q\|_1\\
            &= \|\psi\|_\infty \|\check{\overline{x}}\|_2\|\check{y}\|_2.
            &= \|\psi\|_\infty \|{x}\|_2\|y\|_2.
        \end{align*}
        
        Hence, $\langle Ax,y\rangle \leq \|\psi\|_\infty \|x\|_2\|y\|_2$. So $\|A\|_{\ell^2{\Ntrl}} \leq \|\psi\|_\infty$.
        
        Conversely, suppose that $A$ is bounded on $\ell^2(\Ntrl)$. 
        Note that we must have $a \in \ell^{\infty}(\Ntrl)$, because otherwise
        the sequence 
        \begin{equation*}
            \{\|Ae_n\|\}_{n \geq 0}
        \end{equation*}
        would be unbounded.
        
        Define a linear functional $\omega$ on $\mathcal{P}$ as follows,
        \begin{equation*}
            \omega(p) = \sum_{n\geq 0} a_n \hat{p}(n).
        \end{equation*}
        We wish to show that $\omega$ is bounded in the $L^1(\Circ)$ norm. 
        
        First consider the case that $a \in \ell^1(\Ntrl)$. Then $\check{a} \in C(\Circ)$,
        and
        \begin{equation*}
            \omega(p) = \langle \check{a},\overline{p}\rangle \leq \|\check{a}\|_{\infty}\|p\|_1.
        \end{equation*}
        So for the case $a \in \ell^1(\Ntrl)$, $\omega$ is bounded in the $L^1$ norm,
        and so extends to a continuous linear functional on $H^1$.
        
        For the case $a \notin \ell^1(\Ntrl)$, let $0 < r < 1$ and consider the sequence
        \begin{equation*}
            a^{(r)} := (r^na_n\;:\;n\geq 0)
        \end{equation*}
        and the associated Hankel matrix $A_r = (a^{(r)}_{j+k}\;:\;j,k\geq 0)$.
        If $D_r$ is the diagonal infinite matrix with $j$th diagonal entry $r^j$, then
        $A_r = D_r A D_r$. Since $D_r$ is bounded, we must have $A_r$ is bounded.
        
        Since $a \in \ell^\infty(\Ntrl)$, $a^{r} \in \ell^1(\Ntrl)$. Hence the linear functional
        \begin{equation*}
            \omega_r(p) := \sum_{n\geq 0} a^{(r)}_n\hat{p}(n), p \in \mathcal{P}
        \end{equation*}
        extends continuously to $H^1$.
        
        However, $\omega_r \rightarrow \omega$ for $r\rightarrow 1$ in the weak$^*$ topology (that is,
        for all $\psi \in H^1$, $\omega_r(\psi)\rightarrow \omega(\psi)$).
        
        Hence $\omega$ is bounded on $H^1$ for any $a$ such that $A$ is bounded.
        So $\omega$ extends is of the form $\omega(p) = \langle \psi,p\rangle$
        for some $\psi \in L^\infty$. Hence $a_n = \hat{\psi}(n)$.
        
        
  %      Define a linear function $L$ on the space of sequences of finite support by
  %      \begin{equation*}
  %          L(x_n\;:\;n \geq 0) = \sum_{n\geq 0}a_nx_n.
  %      \end{equation*}
  %      
  %      Suppose that $x \in \ell^1(\Ntrl)$ and $\|x\|_1\leq 1$. Then
  %      we can find $a,b\in \ell^2(\Ntrl)$ such that $x = a*\overline{b}$
  %      and $\|a\|_2 \leq 1$ and $\|b\|_2\leq 1$.
  %      Hence,
  %      \begin{align*}
  %          Lx &= \sum_{n\geq 0}\sum_{j=0}^n a_j\overline{b}_{n-j}\\
  %          &= \sum_{j,k\geq 0}a_{j+k} a_j b_k\\
  %          &= \langle Aa,b\rangle\\
  %      \end{align*}
  %      Hence, $|Lx| \leq \|A\|_{\ell^2(\Ntrl})$ for $\|x\|_1 \leq 1$.
  %      
  %      Hence by the Hahn-Banach theorem, $L$ extends to a linear functional
  %      on $\ell^1(\Ntrl)$.
        
    \end{proof}
    
    We can give another useful characterisation of bounded Hankel operators 
    with the space $\BMOA(\Circ)$, defined as $\BMOA(\Circ) = \BMO(\Circ)\cap H^1$.
    \begin{lemma}
        The Hankel matrix $A = (a_{j+k}\;:\;j,k\geq 0)$ defines a bounded linear operator on $\ell^2(\Ntrl)$
        if and only if the power series
        \begin{equation*}
            \varphi(z) = \sum_{n\geq 0}a_n z^n
        \end{equation*}
        defines an element in $\BMOA(\Circ)$ for $z \in \Circ$.
    \end{lemma}
    \begin{proof}
        If $\varphi \in \BMOA(\Circ)$, then $\varphi = \Proj_+\varphi \in L^\infty(\Circ)$. Hence
        $A$ is bounded.
        
        Now if $A$ is bounded, then there is some $\psi \in L^\infty$ such that $\varphi = \Proj_+\psi = \frac{1}{2}(\psi+\tilde{\psi})$.
        Hence $\varphi \in \BMOA(\Circ)$.
    \end{proof}
    
    \section*{Finite Rank Operators}
        We have characterised those Hankel matrices which define bounded linear operators
        on $\ell^2(\Ntrl)$. We now seek to refine this to the class of finite rank operators
        on $\ell^2(\Ntrl)$.
    

    

    
\end{document}