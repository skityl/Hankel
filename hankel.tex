\documentclass{unswmaths}

\usepackage{unswshortcuts}

\begin{document}

\subject{}
\author{}
\title{}
\studentno{}


\newcommand{\Real}{\operatorname{Re}}
\newcommand{\Img}{\operatorname{Im}}
\newcommand{\lan}{\langle}
\newcommand{\ran}{\rangle}
\newcommand{\Proj}{\mathbb{P}}
\newcommand{\isom}{\cong}
\newcommand{\id}{{\operatorname{id}}}
\newcommand{\ha}{\boldsymbol{m}}
\newcommand{\Circ}{\mathbb{T}}
\newcommand{\BMO}{{BMO}}
\newcommand{\sgn}{\operatorname{sgn}}
\newcommand{\Diff}{\mathcal{D}}
\newcommand{\pvint}{\mathrm{p.v.}\int}

\section*{Introduction}
    A \emph{Hankel matrix} is an infinite matrix of the form $(a_{j+k}\;:\;j,k\geq 0)$ for some 
    sequence $(a_j\;:\;j,\geq 0)$.
    
    Such a matrix defines a linear operator on the space of sequences indexed by $\Ntrl$ of finite support,
    with the matrix $A = (a_{j+k}\;:\;j,k\geq 0)$ acting on the sequence $(x_k\;:\;k\geq 0)$ by matrix
    multiplication,
    \begin{equation*}
        (Ax)_j = \sum_{k\geq 0} a_{j+k}x_k.
    \end{equation*}
    
    These notes cover the conditions under which a Hankel matrix acting in this way extends to a bounded linear
    operator on $\ell^2(\Ntrl)$, and under what conditions this operator is in some ideal of
    compact operators.
    
\section*{Boundedness}
    Suppose that $A = (a_{j+k}\;:\;j,k\geq 0)$ is a Hankel matrix. We wish to find
    conditions on $(a_j\;:\;j\geq 0)$ such that $A$ extends to a bounded operator on $\ell^2(\Ntrl)$. 
    This problem is solved by Nehari % Add reference here 
    as follows:
    \begin{theorem}
        The Hankel matrix $A = (a_{j+k}\;:\;j,k\geq 0)$ extends to a bounded operator on $\ell^2(\Ntrl)$
        if and only if there is some $\psi \in L^\infty(\Circ)$ such that
        \begin{equation*}
            \hat{\psi}(n) = a_n
        \end{equation*}
        for $n\geq 0$.
        Moreover,
        \begin{equation*}
            \|A\|_{\ell^2(\Ntrl)} = \inf\{\|\psi\|_{\infty}\;:\;\hat{\psi}(n) = a_n,n\geq 0\}.
        \end{equation*}
    \end{theorem}
    \begin{proof}
        First assume that $\psi \in L^\infty(\Circ)$ with $\hat{\psi}(n) = a_n$ for $n\geq 0$.
        Suppose that $x = (x_j\;:\;j\geq 0)$ and $y = (y_j\;:\;j\geq 0)$ is a finitely supported sequence. 
        Then we compute,
        \begin{align*}
            \langle Ax,y \rangle &= \sum_{j,k\geq 0} a_{j+k}a_j\overline{y}_k\\
            &= \sum_{j,k\geq 0} \hat{\psi}(j+k)x_j\overline{y}_k\\
            &= \sum_{n \geq 0} \hat{\psi}(n)\sum_{j = 0}^m a_j\overline{y}_{m-j}\\
            &= \sum_{n\geq 0} \hat{\psi}(n) (x * \overline{y})(n)\\
            &= \langle \hat{\psi},\overline{x}*y\rangle
        \end{align*}
        Now let $q = (\overline{x}*y)^{\vee} = \check{\overline{x}} \check{y}$. Hence,
        \begin{align*}
            \langle \hat{\psi},\overline{a}*b\rangle &= \langle \psi,q\rangle\\
            &\leq \|\psi\|_{\infty} \|q\|_1\\
            &= \|\psi\|_\infty \|\check{\overline{x}}\|_2\|\check{y}\|_2.
            &= \|\psi\|_\infty \|{x}\|_2\|y\|_2.
        \end{align*}
        
        Hence, $\langle Ax,y\rangle \leq \|\psi\|_\infty \|x\|_2\|y\|_2$. So $\|A\|_{\ell^2{\Ntrl}} \leq \|\psi\|_\infty$.
        
        Conversely, suppose that $A$ is bounded on $\ell^2(\Ntrl)$. 
        
    \end{proof}
    

    

    
\end{document}